A half single particle model by Group B for the 2023 P\-X915 group project.

\subsection*{Program Features}

S\-P\-A\-M\-S models the charging and discharging of a lithium ion battery using a Crank-\/\-Nicolson semi-\/implicit finite difference scheme to obtain the concentration of lithium in a sphere, {\bfseries c}(i$_{\mbox{app}}$ , {\bfseries r}), at each time step, and has the following features\-:
\begin{DoxyItemize}
\item Apply a constant, stepwise or custom current
\item Options for parallelism
\item Extend the model to a full battery
\item First order sensitivty analysis
\end{DoxyItemize}

\subsection*{Dependencies}

Prior to installing the program or accessing the tutorial, you will need to ensure you have the following installed\-:


\begin{DoxyItemize}
\item intel/2017.\-4.\-196-\/\-G\-C\-C-\/6.\-4.\-0-\/2.\-28
\item impi/2017.\-3.\-196 imkl/2017.\-3.\-196
\item imkl/2017.\-3.\-196
\item net\-C\-D\-F-\/\-Fortran/4.\-4.\-4
\item G\-C\-C/10.\-2.\-0
\item Python/3.\-8.\-6
\item numpy
\item net\-C\-D\-F4
\item matplotlib
\end{DoxyItemize}

For an scrtp managed system the following procedure will ensure correct dependendcies are installed\-:

```bash module purge; module load intel/2017.\-4.\-196-\/\-G\-C\-C-\/6.\-4.\-0-\/2.\-28 impi/2017.\-3.\-196 imkl/2017.\-3.\-196 net\-C\-D\-F-\/\-Fortran/4.\-4.\-4 G\-C\-C/10.\-2.\-0 Python/3.\-8.\-6 pip3 install numpy net\-C\-D\-F4 Matplotlib ```

\subsection*{Installation}

To install S\-P\-A\-M\-S, navigate to the directory in your file system where you would like to download it, and use the following command\-: ```bash git clone \href{https://github.com/HetSys/PX915_GroupB_22-23}{\tt https\-://github.\-com/\-Het\-Sys/\-P\-X915\-\_\-\-Group\-B\-\_\-22-\/23} ```

\subsection*{Running the program}

Run line\-: {\ttfamily python3 \hyperlink{user__input_8py}{user\-\_\-input.\-py}}

\subsection*{Accessing the user tutorial}

A full tutorial for basic usage of the program is provided in Jupyter notebook format.
\begin{DoxyItemize}
\item The notebook can been found in the main directory and is called 'Tutorial.\-ipynb'.
\item The notebook can be viewed here in github as a markdown file.
\item To run the notebook, load it from the terminal by navigating to the main program directory and using the command\-: ```bash nohup jupyter notebook Tutorial.\-ipynb ```
\item You should then be able to find the notebook url using ```bash cat nohup.\-out ```
\end{DoxyItemize}

\subsection*{Developer documentation}

Developer documentation is available at \href{https://hetsys.github.io/PX915_GroupB_22-23/}{\tt https\-://hetsys.\-github.\-io/\-P\-X915\-\_\-\-Group\-B\-\_\-22-\/23/}.



\subsection*{Contributors}

Fraser Birks, Laura Cairns, Sebastian Dooley, Arielle Fitkin, Jake Eller, and Yu Lei

Het\-Sys C\-D\-T, University of Warwick 